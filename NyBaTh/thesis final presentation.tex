% !TEX encoding = UTF-8 Unicode
%\documentclass[handout]{beamer}
\documentclass[bigger]{beamer}
\usetheme{Boadilla}
% !TEX encoding = UTF-8 Unicode
% ÄÖÜ ß äöü



\usepackage[utf8]{inputenc}			% Eingabekodierung
\usepackage[ngerman,english]{babel}	% Unterstütze deutsch und englisch, english ist aktiv, \selectlanguage{ngerman}
\usepackage[T1]{fontenc}				% Korrekte Umlaute im PDF

%\usepackage{ucs}
\usepackage{blindtext}				% Blindtext einfach erzeugen
\usepackage{listings}
\usepackage{floatflt}

%\lstset{
%extendedchars=\true,
%inputencoding=utf8x
%}

\lstset{
extendedchars=\true,
%language=C,                             % Code langugage
basicstyle=\ttfamily,                   % Code font, Examples: \footnotesize, \ttfamily
keywordstyle=\color{OliveGreen},        % Keywords font ('*' = uppercase)
commentstyle=\color{gray},              % Comments font
numbers=left,                           % Line nums position
numberstyle=\tiny,                      % Line-numbers fonts
stepnumber=1,                           % Step between two line-numbers
numbersep=5pt,                          % How far are line-numbers from code
%backgroundcolor=\color{lightlightgray}, % Choose background color
frame=none,                             % A frame around the code
tabsize=2,                              % Default tab size
captionpos=b,                           % Caption-position = bottom
breaklines=true,                        % Automatic line breaking?
breakatwhitespace=false,                % Automatic breaks only at whitespace?
showspaces=false,                       % Dont make spaces visible
showtabs=false,                         % Dont make tabls visible
columns=flexible,                       % Column format
morekeywords={__global__, __device__},  % CUDA specific keywords
}

\usepackage{varioref}
\newcommand{\figref}[1]{(figure \vref{#1})}
\newcommand{\tabref}[1]{(table \vref{#1})}
\newcommand{\seeref}[1]{(see \vref{#1})}
\newcommand{\reff}[1]{„\nameref{#1}“ (see \vref{#1})}
\newcommand{\USD}[1]{{#1}\,\$}

% ############################################################################
% Metadaten und anklickbare Verweise in der PDF-Datei
% ===========================================
%\usepackage[pdftex
%	, pdfauthor={Alexander Maringele}
%	, pdftitle={An Interactive Interface for BoolTool}
%	, pdfsubject={bachaelor thesis},
%		%,pdfkeywords={}
%		%,pdfproducer={Latex with hyperref}
%		%,pdfcreator={pdflatex}
%	, colorlinks=true
%]
%{hyperref}				% anklickbare Links im PDF

% \usepackage[pdftex]{hyperref}

% to typeset algorithms
% \usepackage{algorithm}
% \usepackage{algorithmic}
% for documentation of packages search
% http://ctan.org

\pdfinfo{
/Author (Alexander Maringele)
/Title  (Interactive Interface for Bool Tool)
/CreationDate (D:20121001121300)
/Subject (iPad Learning App for Propositional Logic)
/Keywords (University of Innsbruck;Institute of Computer Science;Computational Logic)
}
% ##############################################################################

\usepackage{graphicx}  % \includgraphics
\usepackage{amssymb}

\usepackage[xindy,toc,nonumberlist]{glossaries}
 % !TEX encoding = UTF-8 Unicode

\newglossaryentry{computer}
{
  name=computer,
  description={is a programmable machine that receives input,
               stores and manipulates data, and provides
               output in a useful format}
}

\newglossaryentry{naiive}
{
  name=na\"{\i}ve,
  description={is a French loanword (adjective, form of naïf)
               indicating having or showing a lack of experience,
               understanding or sophistication}
}


 \makeglossaries


% #########################################################
% smarter line breaks for urls
% =========================================================
%\let\oldurlbreaks=\UrlBreaks
%\renewcommand{\UrlBreaks}{\oldurlbreaks\do\a\do\b\do\c\do\d\do\e%
%  \do\f\do\g\do\h\do\i\do\j\do\k\do\l\do\m\do\n\do\o\do\p\do\q%
%  \do\r\do\s\do\t\do\u\do\v\do\w\do\x\do\y\do\z\do\?\do\&}
%\renewcommand{\UrlBreaks}{\oldurlbreaks\do\-\do\&}
% #########################################################

\usepackage{transparent}
%\usepackage{ngerman}
%\usepackage[utf8]{inputenc} % äöü direkt eintippen
%\usepackage[right]{eurosym}

%\usepackage{subfigure}

%\usepackage{booktabs}

%\usepackage{tabularx}
%\newcolumntype{L}[1]{>{\raggedright\arraybackslash}p{#1}} % linksbündig mit Breitenangabe
%\newcolumntype{C}[1]{>{\centering\arraybackslash}p{#1}} % zentriert mit Breitenangabe
%\newcolumntype{R}[1]{>{\raggedleft\arraybackslash}p{#1}} % rechtsbündig mit Breitenangabe

%\newcommand{\USD}[1]{{#1}\,\$}


 % \setbeamercovered{transparent}

%\usepackage{beamerthemesplit} // Activate for custom appearance

\usebackgroundtemplate{ % \includegraphics[width=3cm]{pics/NyayaAppIcon1024.png}}
\vbox to \paperheight{\vspace{2.3cm}\hbox to \paperwidth{\hfil\includegraphics[width=2.9cm]{pics/NyayaAppIcon1024.png}\hfil}}
}

\title{\Nyaya  for iPad}
\subtitle{Interactive Environment with BoolTool \vspace{1.8cm}}
\author{Alexander Maringele}
\institute[UIBK]{}

\date{March 5th, 2013}

\begin{document}

\frame{
\titlepage
Supervisor: Dr. Georg Moser
}



%\section[Outline]{}

\usebackgroundtemplate{\transparent{0.1}\includegraphics[width=\paperwidth]{fp_images/BoolToolInterface.png}}

\frame {
	\frametitle{What is BoolTool?} 
	\framesubtitle{Manipulation and evaluation\\of formulae in propositional logic}


	
\begin{itemize}
\item<1-5,11> BoolTool is powerful
	\begin{itemize}
	\item<2-5> It defines an input syntax for formulas
	\item<3-5> It derives normal forms
	\item<4-5> It computes truth tables and binary decision diagrams
	\item<5-5> It calculates satisfiability, tautologies and contradictions
	\end{itemize}
\item<1,6-> But it is not for beginners.
	\begin{itemize}
	\item<7-10> It does not motivate or explain anything.
	\item<8-10> It does not use standard symbols of propositional logic
	\item<9-10> It does not explain equivalence transformations
	\item<10-10> It does not define normal forms.
	\end{itemize}
\end{itemize}
	
	}
	
	\usebackgroundtemplate{\transparent{0.1}\includegraphics[width=\paperwidth]{fp_images/SyntaxTreeBackground.png}}

%\frame {
%	\frametitle{BoolTool}
%	\framesubtitle{Web interface}
%%\begin{figure}[htbp]
%%\begin{center}
%\includegraphics[scale=0.35]{fp_images/BoolToolInterface.png}
%%\caption{default}
%%\label{default}
%%\end{center}
%%\end{figure}
%
%}
	
%\frame{
%	\frametitle{BoolTool}
%	\framesubtitle{Drawbacks}
%	
%	\begin{itemize}
%	\item Syntax is slightly different
%	\item Semantics is not explained
%	\item Normal forms are not defined
%	\item Transformations are not demonstrated
%	
%	\end{itemize}
%	
%%	\vfill
%%	
%%	User can not learn many facts about propositional logic.
%}

\frame {
	\frametitle{What is an iPad?}
% a silly question huh? But when you develop for a platform you have to know what you are develop

not:
	useless, an big iphone, the saviour of the newspaper industry

always on, small screen, single window, small ram, small ops
}

\frame {
	\frametitle{What is \Nyaya?}
	\framesubtitle{Project Aim}

	Allow the user to learn

\begin{itemize}
	
	\item Formalism of propositional logic
	
	\item Separation of syntax and semantics
	
	\item Normal forms (NNF, CNF, DNF)

	\item Standard transformations of Boolean functions
	
	\item Coherence of different representations
		
	\end{itemize}
in a self-explanatory environment.

}

%\section{Quellen}



\frame {
	\frametitle{What is \Nyaya?}

	\framesubtitle{Platform agnostic concept}
	
	\begin{itemize}
	
	\item Tutorials for general concepts and definitions

	\item Exercises to consolidate the learned concepts and definitions
	
	\item Playground to build and transform formulas
	
	\item Glossary of technical terms

	\item BoolTool 
	
\end{itemize}
}

\usebackgroundtemplate{\transparent{0.1}\includegraphics[width=\paperwidth]{fp_images/NyayaTabN1.png}}

\frame {
	
	\frametitle{What is \Nyaya?}
	\framesubtitle{Platform specific iPad App}
	%\includegraphics[width=12.2cm,trim=1.5cm 4cm 2cm 2cm]{Images/BoolPad6}
}

\usebackgroundtemplate{\transparent{0.1}\includegraphics[width=\paperwidth]{fp_images/NyayaTabT1.png}}

\frame {
	\frametitle{Tutorials}
}

\usebackgroundtemplate{\transparent{0.1}\includegraphics[width=\paperwidth]{fp_images/NyayaTabE1.png}}
\frame {
	\frametitle{Exercises}
}

\usebackgroundtemplate{\transparent{0.1}\includegraphics[width=\paperwidth]{fp_images/NyayaTabP1.png}}
\frame {
	\frametitle{Playground}
}

\usebackgroundtemplate{\transparent{0.1}\includegraphics[width=\paperwidth]{fp_images/NyayaTabG1.png}}
\frame {
	\frametitle{Glossary}
}

\usebackgroundtemplate{\transparent{0.1}\includegraphics[width=\paperwidth]{fp_images/NyayaTabB1.png}}
\frame {
	\frametitle{BoolTool}
}

\section{Sources}
\frame{
	\frametitle{Sources}
	
	\begin{itemize}
	\item Barwise, Etchemendy und Barker Plummer, \textbf{Tarski's World}
	\item Middeldorp, \textbf{Logic}, Lecture 
	\item Huth and Ryan, \textbf{Logic in Computer Science}
	\item Apple iOS Developer Documentation
	% \item OCaml Sourcecode for BoolTool
	% \item Scofield, \textbf{Cross Compiling OCaml to iOS} %, \href{http://psellos.com/ocaml/compile-to-iphone.html}{www.psellos.com}
	
	\end{itemize}
}

%\section{Tools}
%\frame{
%	\frametitle{Tools}
%	
%	\begin{itemize}
%	\item XCode
%	\item Objective C
%	\item Cocoa
%	
%		
%	\end{itemize}
%}


\end{document}
