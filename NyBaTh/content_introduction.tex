% !TEX encoding = UTF-8 Unicode

\begin{quote}
\em The aim of logic in computer science is to develop languages 
to model the situations we encounter as computer science professionals, 
in such way that we can reason about them formally. \cite{Huth:2004:LCS:975331}
\end{quote}

Logic is used daily. Although the core of reasoning and the foundation of computer science, 
logic still takes time and practice to be mastered soundly and completely. 
Uncertainty in the formalisms of logic leads to fundamental inaccuracies 
in the modeling of various applications in computer science. 
Therefore logic is a mandatory part of every computer science curriculum. 

Each tool that promotes a deeper and more accurate understanding of formalisms and applications of logic,
or simply provides an easier introduction to formal logic, 
is a welcome addition to the toolbox of training.


\paragraph{Propositional logic}is – to put it simply – the entry point and the innermost core of logic. 
It is a powerful formal language, which can be used in many fields. 
But it also has limits of expressiveness and so 
there are many more powerful formal languages of logic, 
which cannot be mastered without a sufficient understanding  of propositional logic.

\paragraph{BoolTool}by the Computational Logic Research Group at the University of Innsbruck allows the manipulation and evaluation of Boolean functions. The tool supports different representations of Boolean functions and a variety of different algorithms.
Propositional logic is one of many representations of Boolean functions.

\paragraph{\Nyaya}“{\em (sanskrit ny-$\bar{\mbox{a}}$yá, literally 'recursion’  used in the sense of  ‘syllogism, inference’) is [...] one of the [...] schools of Hindu philosophy – specifically the school of logic.}”\ \cite{WIKIPEDIA:NYAYA}
%\footnote{\href{http://en.wikipedia.org/wiki/Nyaya}{ en.wikipedia.org/wiki/Nyaya}}. 
“{\em Its followers believed that obtaining valid knowledge was the only way  to obtain release from suffering.}”\ \cite{IEP:NYAYA} 
%\footnote{\href{http://www.iep.utm.edu/nyaya/}{ www.iep.utm.edu/nyaya/}}

\paragraph{\Nyaya for iPad}provides an interactive environment,
that allows the user to learn simple facts about the formalism of propositional logic 
and standard transformations of Boolean functions. 
The environment is self-explanatory, re-implements the functionality of  BoolTool,
provides a graphical editor for syntax trees of propositional formulas, and 
works without a back end on a server.
\Nyaya supports the most effective learning techniques – 
steadily learning and practice testing\ \cite{Dunlosky01012013} –
with its combination 
of small bits of learning content and seemingly countless exercises. 


%\section{Why iPad?}
%
%It‘s popular, it‘s available and has a one defined form factor.
