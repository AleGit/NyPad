% !TEX encoding = UTF-8 Unicode

\begin{quote}
The aim of logic in computer science is to develop languages 
to model the situations we encounter as computer science professionals, 
in such way that we can reason about them formally. \cite{Huth:2004:LCS:975331}
\end{quote}

Logic is used daily. Although the core of reasoning and the foundation of computer science, 
logic still takes time and practice to be mastered soundly and completely. 
Uncertainty in the formalisms of logic leads to fundamental inaccuracies 
in the modeling of various applications in computer science. 
Therefore logic is a mandatory part of every computer science course. 

Each tool that promotes a deeper and more accurate understanding of formalisms and applications of logic,
or simply provides an easier introduction to formal logic, 
is a welcome addition to the toolbox of training.


\section{Propositional logic}

Propositional logic is – to put it simply - the entry point and the innermost core of logic. 
It is a powerful formal language, which can be used in many fields. 
But it also has limits of expressiveness and so 
there are many more powerful formal languages of logic, 
which can not be mastered without a sufficient understanding  of propositional logic.

\section{BoolTool}

BoolTool by the Computational Logic Research Group at the University of Innsbruck allows the manipulation and evaluation of Boolean functions. The tool supports different representations of Boolean functions and a variety of different algorithms.
Propositional logic is one of many representations of Boolean functions.

\section{\Nyaya}

The aim of \Nyaya is to provide an interactive environment on an iPad,
that allows the user to learn simple facts about the formalism of propositional logic 
and standard transformations of Boolean functions. 
The environment is self-explanatory and is integrated with BoolTool. 
\Nyaya supports with its combination 
of small bits of learning content and seemingly countless exercises 
the most effective learning techniques – 
steadily learning and practice testing.\cite{Dunlosky01012013}


%\section{Why iPad?}
%
%It‘s popular, it‘s available and has a one defined form factor.
