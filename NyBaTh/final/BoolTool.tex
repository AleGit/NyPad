% !TEX encoding = UTF-8 Unicode
%

\frame {
	\frametitle{BoolTool} 
	\framesubtitle{Manipulation and evaluation\\of formulas in propositional logic}


	
\begin{itemize}
\item<1-5,11> BoolTool is powerful
	\begin{itemize}
	\item<2-5> It defines an input syntax for formulas
	\item<3-5> It derives normal forms
	\item<4-5> It computes truth tables and binary decision diagrams
	\item<5-5> It calculates satisfiability, tautologies and contradictions
	\end{itemize}
\item<1,6-> But it is not for beginners.
	\begin{itemize}
	\item<7-10> It does not motivate or explain much.
	\item<8-10> It does not use standard symbols of propositional logic.
	\item<9-10> It does not explain equivalence transformations.
	\item<10-10> It does not define normal forms.
	\end{itemize}
\end{itemize}
	
}