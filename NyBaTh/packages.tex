% !TEX encoding = UTF-8 Unicode
% ÄÖÜ ß äöü

\usepackage[utf8]{inputenc}			% Eingabekodierung
\usepackage[ngerman,english]{babel}	% Unterstütze deutsch und englisch, english ist aktiv, \selectlanguage{ngerman}
\usepackage[T1]{fontenc}				% Korrekte Umlaute im PDF

%\usepackage{ucs}
\usepackage{blindtext}				% Blindtext einfach erzeugen
\usepackage{listings}
\usepackage{floatflt}
\usepackage{color}

\lstset{
extendedchars=\true,
language=XML,                           % PLIST is pure XML
basicstyle=\ttfamily,                   % Code font, Examples: \footnotesize, \ttfamily
keywordstyle=\color{blue},              % Keywords font ('*' = uppercase)
commentstyle=\color{red},               % Comments font
numbers=left,                           % Line nums position
numberstyle=\tiny,                      % Line-numbers fonts
stepnumber=1,                           % Step between two line-numbers
numbersep=5pt,                          % How far are line-numbers from code
%backgroundcolor=\color{lightlightgray}, % Choose background color
frame=none,                             % A frame around the code
tabsize=2,                              % Default tab size
captionpos=b,                           % Caption-position = bottom
breaklines=true,                        % Automatic line breaking?
breakatwhitespace=false,                % Automatic breaks only at whitespace?
showspaces=false,                       % Dont make spaces visible
showtabs=false,                         % Dont make tabls visible
columns=flexible,                       % Column format
morekeywords={array,dict,string,integer,key,date,data},     % PLIST specific keywords
,mathescape=true
}

\usepackage{varioref}
\newcommand{\Figref}[1]{(Figure \vref{#1})}
\newcommand{\Tabref}[1]{(Table \vref{#1})}
\newcommand{\tabref}[1]{table \vref{#1}}
\newcommand{\figref}[1]{figure \vref{#1}}
\newcommand{\seeref}[1]{(see \vref{#1})}
\newcommand{\reff}[1]{„\nameref{#1}“ (see \vref{#1})}
\newcommand{\USD}[1]{{#1}\,\$}
\newcommand{\TAP}[1]{\;{\boldsymbol{\color{magenta}#1}}\;}

\newcommand{\UML}[1]{\includegraphics[scale=0.65, trim=0.2cm 0.6cm 0 0, clip=true]{uml/Nyaya/#1}}

% ############################################################################
% Metadaten und anklickbare Verweise in der PDF-Datei
% ===========================================
%\usepackage[pdftex
%	, pdfauthor={Alexander Maringele}
%	, pdftitle={An Interactive Interface for BoolTool}
%	, pdfsubject={bachaelor thesis},
%		%,pdfkeywords={}
%		%,pdfproducer={Latex with hyperref}
%		%,pdfcreator={pdflatex}
%	, colorlinks=true
%]
%{hyperref}				% anklickbare Links im PDF

% \usepackage[pdftex]{hyperref}

% to typeset algorithms
% \usepackage{algorithm}
% \usepackage{algorithmic}
% for documentation of packages search
% http://ctan.org

\pdfinfo{
/Author (Alexander Maringele)
/Title  (Interactive Interface for Bool Tool)
/CreationDate (D:20121001121300)
/Subject (iPad Learning App for Propositional Logic)
/Keywords (University of Innsbruck;Institute of Computer Science;Computational Logic)
}
% ##############################################################################

\usepackage{graphicx}  % \includgraphics
\usepackage{amssymb}

\usepackage[xindy,toc,nonumberlist]{glossaries}
 % !TEX encoding = UTF-8 Unicode

\newglossaryentry{computer}
{
  name=computer,
  description={is a programmable machine that receives input,
               stores and manipulates data, and provides
               output in a useful format}
}

\newglossaryentry{naiive}
{
  name=na\"{\i}ve,
  description={is a French loanword (adjective, form of naïf)
               indicating having or showing a lack of experience,
               understanding or sophistication}
}


 \makeglossaries


% #########################################################
% smarter line breaks for urls
% =========================================================
%\let\oldurlbreaks=\UrlBreaks
%\renewcommand{\UrlBreaks}{\oldurlbreaks\do\a\do\b\do\c\do\d\do\e%
%  \do\f\do\g\do\h\do\i\do\j\do\k\do\l\do\m\do\n\do\o\do\p\do\q%
%  \do\r\do\s\do\t\do\u\do\v\do\w\do\x\do\y\do\z\do\?\do\&}
%\renewcommand{\UrlBreaks}{\oldurlbreaks\do\-\do\&}
% #########################################################
