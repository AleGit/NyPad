% !TEX encoding = UTF-8 Unicode

%% LaTeX2e file `appendix.tex'
%% generated by the `filecontents' environment
%% from source `CLaTeX' on 2012/04/26.
%%


% \chapter{Glossary}

%\gls{naiive} \gls{computer}





\chapter{Handbook}

\chapter{Definitions}

\begin{table}[htdp]
\begin{center}
\begin{lstlisting}[mathescape]
<!ENTITY % plistObject "(array | data | date | dict | real | integer | string | true | false )" >
<!ELEMENT plist %plistObject;>
<!ATTLIST plist version CDATA "1.0" >

<!-- Collections -->
<!ELEMENT array (%plistObject;)*>
<!ELEMENT dict (key, %plistObject;)*>
<!ELEMENT key (#PCDATA)>

<!--- Primitive types -->
<!ELEMENT string (#PCDATA)>
<!ELEMENT data (#PCDATA)> <!-- Contents interpreted as Base-64 encoded -->
<!ELEMENT date (#PCDATA)> <!-- Contents should conform to a subset of ISO 8601 (in particular, YYYY '-' MM '-' DD 'T' HH ':' MM ':' SS 'Z'.  Smaller units may be omitted with a loss of precision) -->

<!-- Numerical primitives -->
<!ELEMENT true EMPTY>  <!-- Boolean constant true -->
<!ELEMENT false EMPTY> <!-- Boolean constant false -->
<!ELEMENT real (#PCDATA)> <!-- Contents should represent a floating point number matching ("+" | "-")? d+ ("."d*)? ("E" ("+" | "-") d+)? where d is a digit 0-9.  -->
<!ELEMENT integer (#PCDATA)> <!-- Contents should represent a (possibly signed) integer number in base 10 -->
\end{lstlisting}
\caption{\href{http://www.apple.com/DTDs/PropertyList-1.0.dtd}{PropertyList-1.0.dtd}}
\label{tab:PLISTDTD}
\end{center}

\end{table}%

\begin{table}[htdp]
\begin{center}
\begin{tabular}{llll}
&&&Tutorials.plist \\
\hline
\hline
%& Tutorial & Instructions & Configuration \\
\hyperref[tut:101]{101} & tutorial101.html & instructions101.html & 101.plist \\
\hyperref[tut:1DS]{1DS} & tutorial1DS.html & instructions1DS.html & 1DS.plist \\
\hyperref[tut:1SY]{1SY} & tutorial1DS.html & instructions1DS.html & 1SY.plist \\
\hyperref[tut:1LI]{1LI} & tutorial1LI.html & & \\
\hline
\hyperref[tut:201]{201} & \\
\hyperref[tut:2PRE]{2PRE} & \\
\hyperref[tut:2SUB]{2SUB} & \\
\hyperref[tut:2SYT]{2SYT} & \\
\hyperref[tut:2TOB]{2TOB} & \\
\hyperref[tut:2SMP]{2SMP} & \\
\hyperref[tut:2CNF]{2CNF} & \\
\hyperref[tut:2DNF]{2DNF} & \\
\hline
\hyperref[tut:301]{301} & \\
\hyperref[tut:3TT]{3TT} & \\
\hyperref[tut:3EE]{3EE} & \\
\hyperref[tut:3STC]{3STC} & \\
\hline
\hyperref[tut:401]{401} & \\
\hyperref[tut:4IFF]{4IFF} & \\
\hyperref[tut:4NNF]{4NNF} & \\
\hyperref[tut:4CNF]{4CNF} & \\
\hyperref[tut:4DNF]{4DNF} & \\
\hyperref[tut:4ALG]{4ALG} & \\
\hyperref[tut:4VAL]{4VAL} & \\
\hyperref[tut:4SAT]{4SAT} & \\
\hline
\hyperref[tut:501]{501} & tutorial501.html \\
%\hyperref[tut:]{} & \\
%\hyperref[tut:]{} & \\
%\hyperref[tut:]{} & \\
%\hyperref[tut:]{} & \\
%\hyperref[tut:]{} & \\
%\hyperref[tut:]{} & \\
%\hyperref[tut:]{} & \\
%\hyperref[tut:]{} & \\
%\hyperref[tut:]{} & \\
%\hyperref[tut:]{} & \\
%\hyperref[tut:]{} & \\
\end{tabular}
\caption{Overview of all configuration and content files}
\label{tab:CONFIG}
\end{center}
\end{table}%
