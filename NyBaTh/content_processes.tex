% !TEX encoding = UTF-8 Unicode


\section{Tools}

\subsection{Developing for the iPad}

A computer with Mac OS X and 
\href{sec:Xcode}{Xcode} are needed for first steps in developing software for iPad. 
Knowledge of object-oriented programming, object-oriented design patterns, 
and the C programming language 
builds a good base to learn the main platform-specific techniques 
\href{sec:ObjC}{Objective-C},
\href{sec:Cocoa}{Cocoa}, and
\href{sec:MemoryManagement}{Memory Management}
with automatic reference counting.

In addition to the technical skills, it is to gain an overview to the necessary steps for 
\href{sec:DAD}{Development and Distribution} (p.~\pageref{sec:DAD}) for the iPad platform.

\subsubsection{Xcode}
\label{sec:Xcode}

Xcode is Apple's integrated development environment to create software for OS X and iOS devices 
.
Xcode is free to download. It includes an iPad simulator, complete API documentations and development guides to various topics. Xcode supports the source management systems git and subversion. 
Git will be installed along various other development tools with Xcode.

\subsubsection{Objective-C}
\label{sec:ObjC}
Objective-C is a reflective, object-oriented extension of the C programming language,
 which was developed by Tom Love and Brad Cox in the early 1980s (a few moments earlier than C++). 
The syntax for objects and methods is based on Smalltalk. 
Objective-C method calls will be bound to functions at runtime with no need to be defined formally at compile time.

\subsubsection{Cocoa Touch}
\label{sec:Cocoa}

Cocoa Touch is the name for the object-oriented APIs for iOS (the operating system for iPhone and iPad). 
Cocoa Touch covers the platform specific Objective-C runtime and a platform specific set of libraries, the frameworks.

\begin{itemize}
\item The Foundation framework it provides the basis for programming with Objective-C. 
In addition to the memory and exception handling it includes the base classes for strings, values, lists, sets and files.
\item The UIKit framework provides a set of classes and functions for implementing (touch based) graphical user interfaces. 
The architecture of this framework follows the Model-View-Controller pattern 
and the implementation provides a myriad of view and controller classes.
\end{itemize}

\subsubsection{Memory Management}
\label{sec:MemoryManagement}
The Objective-C runtime does not offer – surprisingly for Java or C\# developers  – garbage collection, 
but dynamic memory management based on reference counting. 
The necessary retains and releases are added automatically at compile time.
This has the consequence that (strong) object graphs must be implemented as directed acyclic graphs. 
Back references have to be declared as weak.
Otherwise objects would never be released. 

The advantages of reference counting in comparison to automatic garbage collection 
are the deterministic and economical run-time behavior.
Objects are destroyed at a defined time and a separate thread for garbage collection is not necessary. 

%These were very important features in 2007, when the original iPhone was introduced, because
%the hardware for mobile devices was very limited, but looses importance every year.

\subsubsection{Development and distribution}
\label{sec:DAD}
Development and distribution for Apple's platforms includes the coding effort and expenses for administration and configuration. 
To install software on iOS devices – especially for distribution via the App Store – applications must be cryptographically signed. 
The necessary certificates are created in the paid members section on Apple's Developer Portal. 
Certificates, development and distribution profiles contain a bunch of identifiers for the identification and differentiation of developers (Team ID), 
applications (App-ID) and a list of permissions (entitlements). 

\subsection{Source Code Management}
\label{sec:SCM}




\subsubsection{Git}

Git \cite{Git:Main} is a distributed source control management and revision control system 
- originally developed by Linus Torvalds for Linux kernel development. 
It supports local repositories with full source code history for development 
and the synchronization and merging with remote repositories for collaboration or backup. 
\cite{Chacon:2009:PG:1618548} 

Since Apple's IDE Xcode supports Git's local repositories out of the box it was obvious to use git as source control.

\subsubsection{Github}

GitHub \cite{GitHub:Main} is hosting service for projects that use Git. 
GitHub offers free accounts for public repositories, 
which can be seen and downloaded by anyone
using a web-browser or git.
Committing changes is restricted to users chosen by the owner of the repository.


As Nyaya is a student project and the source code should be publicly accessible,
GitHub seems to be suitable to house the project.

\section{Project execution}

\subsection{Phases}

\subsubsection{Fooling around – Views}

Spring 2012 – Playing with the visualization of syntax trees and different kinds of user interaction, informal specification of the feature set

\subsubsection{Getting seriously – Model}

Summer 2012 – Development of the Concept, Include Parser, data structure for AST, BDD

\subsubsection{Make it usable – Controllers and Content}

Autumn 2012 – 

\subsection{Development model}

The development of this software did no follow a specific development model, 
but did borrow some principles from agile development.

\subsubsection{Test-driven development}



\subsubsection{Refactoring}

\subsubsection{Use cases}


