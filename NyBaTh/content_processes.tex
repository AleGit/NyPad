% !TEX encoding = UTF-8 Unicode


\section{Tools}

\subsection{Developing for the iPad}

For the first steps in developing software for iPad
a computer with Mac OS X and \href{sec:Xcode}{Xcode} (p.~\pageref{sec:Xcode}) is needed. 

Knowledge of object-oriented programming, object-oriented design patterns, and the C programming language 
builds a good base to learn the main platform-specific techniques 
\href{sec:ObjC}{Objective-C} (p.~\pageref{sec:ObjC}),
\href{sec:Cocoa}{Cocoa} (p.~\pageref{sec:Cocoa}), and
\href{sec:MemoryManagement}{Memory Management} (p.~\pageref{sec:MemoryManagement}).

In addition to the technical skills, it is to gain an overview to the necessary steps for 
\href{sec:DAD}{Development and Distribution} (p.~\pageref{sec:DAD}) for the iPad platform.

\subsubsection{Xcode}
\label{sec:Xcode}

Xcode is Apple's integrated development environment to create software for OS X and iOS devices.
Xcode is free to download. It includes an iPad simulator, complete API documentations and development guides to various topics.

\subsubsection{Objective-C}
\label{sec:ObjC}
Objective-C is a reflective, object-oriented extension of the C programming language,
 which was developed by Tom Love and Brad Cox in the early 1980s (a few moments earlier than C++). 
The syntax for objects and messages was based on Smalltalk. 
Objective-C does not have method calls, but messages, which are bound to classes and instances at runtime.

\subsubsection{Cocoa}
\label{sec:Cocoa}

Cocoa is the name for the object-oriented APIs for iOS (the operating system for iPhone and iPad). 
Cocoa covers the Objective-C runtime and a set of libraries, the frameworks.
\begin{itemize}
\item The Foundation framework it provides the basis for programming with Objective-C. 
In addition to the memory and exception handling it includes the base classes for strings, values, lists, sets and files.
\item The UIKit framework provides a set of classes and functions for implementing graphical user interfaces. 
The architecture of this framework follows the Model-View-Controller pattern 
and the implementation provides a myriad of view and controller classes.
\end{itemize}

\subsubsection{Memory Management}
\label{sec:MemoryManagement}
The runtime of Objective-C does not offer – surprisingly especially for pure Java or C\# developers  – no garbage collection, 
but an reference counting dynamic memory management. The necessary retains and releases are added automatically at compile time.
This has the consequence that (strong) object graphs must be implemented as directed acyclic graphs. Back references have to be declared as weak.
Otherwise objects would be never released. 

The advantages of the reference counting in comparison to automatic garbage collection are in the foreseeable and economical run-time behavior.
Objects are destroyed at a defined time. A separate thread for garbage collection is not necessary. 

%These were very important features in 2007, when the original iPhone was introduced, because
%the hardware for mobile devices was very limited, but looses importance every year.

\subsubsection{Development and distribution}
\label{sec:DAD}
Development and distribution for Apple's platforms includes the coding effort and expenses for administration and configuration. 
To install software on iOS devices – especially for distribution via the App Store – applications must be cryptographically signed. 
The necessary certificates are created in the paid members section on Apple's Developer Portal. 
Certificates, development and distribution profiles contain a bunch of identifiers for the identification and differentiation of developers (Team ID), 
applications (App-ID) and lists of permissions (entitlements). 

\subsection{Version Control}

\subsubsection{Git}

\subsubsection{Github}

\section{Project management}

\subsection{Phases}

\subsubsection{Fooling around – Views}

Spring 2012 – Playing with the visualization of syntax trees and different kinds of user interaction, informal specification of the feature set

\subsubsection{Get serious – Model}

Summer 2012 – Development of the Concept, Include Parser, data structure for AST, BDD

\subsubsection{Make it usable – Controllers}

Autumn 2012 – 

\subsection{Development model}

The development of this software did no follow a specific development model, 
but did borrow some principles from agile development.

\subsubsection{Test-driven development}

\subsubsection{Refactoring}

\subsubsection{Use cases}
