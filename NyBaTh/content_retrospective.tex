% !TEX encoding = UTF-8 Unicode
% ÄÖÜ ß äöü

The first version of \Nyaya is a fully functional eLearning App, 
which covers it’s purpose well.
But it is still far from perfect.
Some features have to be re-engineered before additional features can be implemented.

\section{Important improvements}

Although some technical aspects of \Nyaya are working correctly, they need a revision, 
because they are too inflexible or too slow.

\begin{itemize}

\item At this time content updates can only be done with an application update in the App Store,
which will lead to some delay in the distribution of the correction of typing errors or an update of the glossary.

\item Syntax trees are stored memory optimized as acyclic syntax diagrams. 
Nevertheless the valuation does not use this fact and the same sub-tree will be evaluated multiple times.

\item The creation of binary decision diagrams is very slow for formulas with more than 15 different atoms.

\end{itemize}

Some aspects of \Nyaya’s user interface does not use the full richness of Cocoa touch. 
Others lack the graphical sophistication that users are accustomed on the iOS platform.

\begin{itemize}

\item Accessibility features of iOS have to be supported – especially 
\verb+VoiceOver+, \verb+Zoom+, \verb+LargeText+ and \verb+AssistiveTouch+,
which will enable \Nyaya to reach a wider audience.

\item Although English is the technical language of computer science,
localizations of the user interface, the tutorials and the glossary will increase 
the reachable audience too.
%
%%\begin{itemize}
%%\item Most European languages should be covered eventually. 
%%\item Arabic and Hebrew are written left to right
%%\item East-Asian languages will produce new challenges.
%%\item 
%%\end{itemize}
%


\item Content and user interface should be presented in a visually more appealing manner.
For the first css-definitions should be developed, for the second additional images have to be created.

\end{itemize}
%
%\section{Outlook)
%
%\begin{itemize}
%\item iPhone
%
%\item Android
%
%\item Maco
%
%\end{itemize}

\section{Outlook}

To increase the number of potential users
\Nyaya could be extended with new feature
or \Nyaya could be ported to other platforms.

\subsection{Additional features}

Even it is unlikely the following features will be implemented in the foreseeable future, 
the following use cases would be obvious evolutions for \Nyaya.

\begin{itemize}

\item Creation and editing of arbitrary binary decision diagrams similar to abstract syntax trees.

\item Definition of arbitrary Boolean functions with truth tables, propositional formulas, abstract syntax trees or binary decision diagrams.

\end{itemize}


\subsection{Additional platforms}

The author of Ny$\bar{a}$a is an apologist of native client applications,
although he admits that they are many useful use cases for modern web-applications, 
using html5, css3 and JavaScript. 
Unfortunately the freely available development tool chains for (graphical) html/javascript-applications
are very limited and far from consolidated, although the underlining standards html5, css3 
and JavaScript 1.8 would provide anything needed, the workload to create sophisticated web-apps 
is much higher than to develop native apps on the same level.

So the most likely platforms to be supported by \Nyaya are Mac OS X and Android.

\begin{itemize}
\item
For a Mac OS X version some aspects of the user interface must be re-thought and re-implemented, 
because Mac OS X does not support a touch interface.
\item
For an Android version the model must be translated into Java and the user interface must be re-implemented.
\end{itemize}

