% !TEX encoding = UTF-8 Unicode
% ÄÖÜ ß äöü

The first version of Ny$\bar{a}$ya is a fully functional eLearning App, 
which covers it’s purpose well.
But it is still far from perfect.
Some features have to be re-engineered before additional features can be implemented.

\section{Important improvements}

It is planned to implement the following features 

\subsubsection{Technical}

\begin{itemize}

\item Content updates from a web-server 
with no need to update the Ny$\bar{a}$ya application in the App Store 
for every small correction of a typing error. 

\item Speed up the creation of binary decision diagrams

\item Optimizations in valuations and transformations

\end{itemize}

\subsubsection{User Interface}

\begin{itemize}

\item Accessibility features of iOS should be supported too – especially 
\verb+VoiceOver+, \verb+Zoom+, \verb+LargeText+ and \verb+AssistiveTouch+.

\item Although English is the technical language of computer science,
a translation of the tutorials into as many languages as possible will increase the practical value of Ny$\bar{a}$ya.
Ny$\bar{a}$ya will support English, Spanish, German, French, Italian, Czech, Slovakian, Hungarian, Croatian, Slovenian, Romanian and Turkish
until the end of 2013.
%
%%\begin{itemize}
%%\item Most European languages should be covered eventually. 
%%\item Arabic and Hebrew are written left to right
%%\item East-Asian languages will produce new challenges.
%%\item 
%%\end{itemize}
%


\item Both content and user interface should be presented in a visually more appealing manner.

\end{itemize}
%
%\section{Outlook)
%
%\begin{itemize}
%\item iPhone
%
%\item Android
%
%\item Maco
%
%\end{itemize}

\section{Outlook}

Even it is unlikely that any of the following features will be implemented in the foreseeable future, 
the following features are obvious evolutions, that would broaden the scope of Ny$\bar{a}$ya.

\subsection{Additional features}

\begin{itemize}

\item Creation and editing of arbitrary binary decision diagrams similar to abstract syntax trees.

\item Definition of arbitrary Boolean functions with truth tables, propositional formulas, abstract syntax trees or binary decision diagrams.

\end{itemize}


\subsection{Platforms}
\begin{itemize}

\item For a Mac OS X version some aspects of the user interface must be re-thought and re-implemented, 
because Mac OS X does not support a touch interface.

\item For a Android version the model must be translated into Java and the user interface must be re-implemented.



\end{itemize}




